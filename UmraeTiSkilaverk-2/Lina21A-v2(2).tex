\documentclass[12pt]{article}

\usepackage[T1]{fontenc} \usepackage[icelandic]{babel}
\usepackage{latexsym,amssymb,amsmath}
\usepackage[utf8]{inputenc}

\voffset=-1.0in
\hoffset=-0.5in
\textwidth=6in
\textheight=10.0in

\begin{document}

\pagestyle{empty}


\newcommand{\Ov}{\mbox{${\bf 0}$}}
\newcommand{\rv}{\mbox{${\bf r}$}}
\newcommand{\vv}{\mbox{${\bf v}$}}
\newcommand{\xv}{\mbox{${\bf x}$}}
\newcommand{\yv}{\mbox{${\bf y}$}}
\newcommand{\av}{\mbox{${\bf a}$}}
\newcommand{\bv}{\mbox{${\bf b}$}}
\newcommand{\cv}{\mbox{${\bf c}$}}
\newcommand{\iv}{\mbox{${\bf i}$}}
\newcommand{\jv}{\mbox{${\bf j}$}}
\newcommand{\kv}{\mbox{${\bf k}$}}

\newcommand{\ZZ}{\mbox{$\mathbb{Z}$}}
\newcommand{\NN}{\mbox{$\mathbb{N}$}}
\newcommand{\QQ}{\mbox{$\mathbb{Q}$}}
\newcommand{\RR}{\mbox{$\mathbb{R}$}}
\newcommand{\CC}{\mbox{$\mathbb{C}$}}

\newcommand{\RRk}{\mbox{$\mathbb{R}^k$}}
\newcommand{\RRn}{\mbox{$\mathbb{R}^n$}}
\newcommand{\RRm}{\mbox{$\mathbb{R}^m$}}

\centerline{\bf \Huge Línuleg algebra A}
%\centerline{(STÆ106G)}
\medskip
\centerline{\bf \Large Verkefnablað vegna umræðutíma 2}

\medskip



{\bf Torfi Þorgrímsson}


%\smallskip


\medskip

{\bf Verkefni fyrir umræðutíma 20.09.21:}  

\smallskip

{\bf 1.}  Setjum $A=\{x+y\sqrt{2}\mid x,y\in \QQ\}\subseteq\RR$.  

Sýnið að ef $x,y\in\QQ$ þá er $x+y\sqrt{2}=0$ ef og aðeins ef $x=y=0$.

Sýnið að ef $z_1, z_2\in A$ þá er $z_1+z_2, z_1z_2\in A$.

\smallskip
  
{\bf 2.}  Úr dæmakafla bls.~57 í bók Donaldson og Pantano:  4.1.1, 4.1.2, 4.1.7. 

\bigskip

{\bf Verkefni fyrir umræðutíma 27.09.21:}  

\smallskip

{\bf 1.}  Úr dæmakafla bls.~59--60 í bók Donaldson og Pantano:  4.2.1, 4.2.3, 4.2.4.  

\smallskip

{\bf 2.}  Úr dæmakafla bls.~64--65 í bók Donaldson og Pantano:  4.3.2, 4.3.5, 4.3.6 (c)(d). 


\bigskip

{\bf Verkefni fyrir umræðutíma 04.10.21:} 

\smallskip

{\bf 1.}  Úr dæmakafla bls.~99--100 í bók Donaldson og Pantano:  6.1.1, 6.1.2, 6.1.3, 6.1.6.

\smallskip

{\bf 2.}  Úr dæmakafla bls.~104--105 í bók Donaldson og Pantano:  6.2.1 (a)(d), 6.2.3, 6.2.6.


\smallskip

%{\bf Verkefni fyrir umræðutíma 11.10.17:} 

%\smallskip

%{\bf 1.}  Úr dæmakafla bls.~104--105 í bók Donaldson og Pantano: 6.2.4. 
 
%\smallskip

%{\bf 2.}  Úr dæmakafla bls.~72--74 í bók Donaldson og Pantano: 4.4.1, 4.4.4, 4.4.6, 4.4.7, 4.4.8, 4.4.9.  

\bigskip
{\bf Skilverkefni fyrir 29.09.21:}  Þessi verkefni eru einstaklingsverkefni.  Lausnum verkefnanna á að skila miðvikudaginn 29.~september  í gegnum Canvaskerfið.

\smallskip

{\bf 1.}  Látum $A$, $B$ og $C$ vera mengi.  

a.  Sannið eða afsannið:  Ef $A\setminus C\subseteq B\setminus C$ þá er $A\subseteq B$.

b.  Sannið eða afsannið:  Ef $A\subseteq B$ þá er $A\setminus C\subseteq B\setminus C$.

\smallskip

{\bf 2.}  Verkefni 4.3.3 á bls.~65

\smallskip

{\bf 3.}  Sannið formlega að ef $A, B$ og $C$ er mengi þá er $A\cap (B\cup C)=(A\cap B)\cup (A\cap C)$.

\smallskip

{\bf 4.}  Verkefni 6.2.4 og 6.2.5 á bls.~104--105.



\vfill  17.~september 2021
\hfill  Rögnvaldur G. Möller


\end{document}