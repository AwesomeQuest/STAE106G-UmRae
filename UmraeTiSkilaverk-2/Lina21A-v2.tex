\documentclass[12pt]{article}

\usepackage[T1]{fontenc} \usepackage[icelandic]{babel}
\usepackage{latexsym,amssymb,amsmath}
\usepackage[utf8]{inputenc}
\usepackage{enumitem}

\voffset=-1.0in
\hoffset=-0.5in
\textwidth=6in
\textheight=10.0in

\begin{document}

\pagestyle{empty}


\newcommand{\Ov}{\mbox{${\bf 0}$}}
\newcommand{\rv}{\mbox{${\bf r}$}}
\newcommand{\vv}{\mbox{${\bf v}$}}
\newcommand{\xv}{\mbox{${\bf x}$}}
\newcommand{\yv}{\mbox{${\bf y}$}}
\newcommand{\av}{\mbox{${\bf a}$}}
\newcommand{\bv}{\mbox{${\bf b}$}}
\newcommand{\cv}{\mbox{${\bf c}$}}
\newcommand{\iv}{\mbox{${\bf i}$}}
\newcommand{\jv}{\mbox{${\bf j}$}}
\newcommand{\kv}{\mbox{${\bf k}$}}

\newcommand{\ZZ}{\mbox{$\mathbb{Z}$}}
\newcommand{\NN}{\mbox{$\mathbb{N}$}}
\newcommand{\QQ}{\mbox{$\mathbb{Q}$}}
\newcommand{\RR}{\mbox{$\mathbb{R}$}}
\newcommand{\CC}{\mbox{$\mathbb{C}$}}

\newcommand{\RRk}{\mbox{$\mathbb{R}^k$}}
\newcommand{\RRn}{\mbox{$\mathbb{R}^n$}}
\newcommand{\RRm}{\mbox{$\mathbb{R}^m$}}

\centerline{\bf \Huge Línuleg algebra A}
%\centerline{(STÆ106G)}
\medskip
\centerline{\bf \Large Verkefnablað vegna umræðutíma 2}

\medskip

\bigskip
{\bf Skilverkefni fyrir 29.09.21:}  Þessi verkefni eru einstaklingsverkefni.  Lausnum verkefnanna á að skila miðvikudaginn 29.~september  í gegnum Canvaskerfið.

\smallskip

{\bf 1.}  Látum $A$, $B$ og $C$ vera mengi.  

\begin{enumerate}[label=(\alph*)]
	\item Sannið eða afsannið:  Ef $A\setminus C\subseteq B\setminus C$ þá er $A\subseteq B$.
	\item Sannið eða afsannið:  Ef $A\subseteq B$ þá er $A\setminus C\subseteq B\setminus C$.
\end{enumerate}

\smallskip

{\bf 2.}  Verkefni 4.3.3 á bls.~65

Let $A\subseteq \mathbb{R}$, and let $x\in \mathbb{R}$. We say that the point $x$ is \emph{far away} from the set A if and only if:

\qquad \qquad $\exists d >0$: No element of $A$ belongs to the set $[x-d,x]$.

Equivalently, $A\cap [x-d,x]=\emptyset $. if this does not happen, we say that $x$ is \emph{close} to $A$

\begin{enumerate}[label=(\alph*),itemsep=0pt]
	\item Draw a  picture of a set $A$ and an element $x$ such that $x$ is \emph{far away} from $A$.
	\item Draw a  picture of a set $A$ and an element $x$ such that $x$ is \emph{close} to $A$.
	\item Compute the definition of "$x$ is \emph{close} to $A$". [So negate "$x$ is far away from $A$".]
	\item Let $A=\{ 1,2,3\} $. Show that $x=4$ is \emph{far away} from $A$ by using definitions.
	\item Let $A=\{ 1,2,3\} $. Show that $x=1$ is \emph{close} to $A$ by using definitions.
	\item Show that if $x\in A$, then $x$ is \emph{close} to $A$.
	\item Let $A$ be the open interval $(a,b)$. Is the end-point $a$ \emph{far away} from $A$? What about the end point $b$
\end{enumerate}

\smallskip

{\bf 3.}  Sannið formlega að ef $A, B$ og $C$ er mengi þá er $A\cap (B\cup C)=(A\cap B)\cup (A\cap C)$.

\smallskip

{\bf 4.}  Verkefni 6.2.4 og 6.2.5 á bls.~104--105.

Here are three incorrect proofs of Theorem 6.5. Explain why each fails.

\begin{enumerate}[label=(\alph*),itemsep=0pt]
	\item Let $x=\mathcal{P} (A)$. Then $x\in A$. Since $A\subseteq B$, we have $x\in B$. Therefore $x\in \mathcal{P} (B)$, and so $\mathcal{P} (A) \subseteq \mathcal{P} (B)$.
	\item Let $A=\{ 1,2\} $ and $B=\{ 1,2,3\} $. Then $\mathcal{P}(A)=\{ \emptyset , \{ 1\} ,\{ 2\} ,A\} $, and $\mathcal{P}(B)=\{ \emptyset ,\{ 1\} ,\{ 2\} ,\{ 3\} ,\{ 1,2\} ,\{ 1,3\} ,\{ 2,3\} ,B\} $. Thus $\mathcal{P}(A)\subseteq \mathcal{P}(B)$.
	\item Let $x\in A$. Since $A\subseteq B$, we have $x\in B$. Since $x\in A$ and $x\in B$, we have $\{ x\} \in \mathcal{P}(A)$, and $\{ x\} \in \mathcal{P}(B)$.
\end{enumerate}

\vfill  17.~september 2021
\hfill  Rögnvaldur G. Möller


\end{document}